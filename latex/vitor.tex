
\title{Projeto de Pesquisa para Doutorado}
\label{projeto-de-pesquisa-para-doutorado}

\author{Vitor Brandão Sabbagh}

\date{Novembro de 2025}

\maketitle

%
\section{Título}\label{tuxedtulo} 



\textbf{Mapeando a Fronteira Irregular: Uma Análise Experimental da
Capacidade de Agentes Baseados em LLM em Tarefas de Engenharia de Poços}

\noindent
\emph{(Mapping the Jagged Frontier: An Experimental Analysis of
LLM-Based Agent Capabilities in Complex Offshore Well Engineering
Tasks)}

\wagner{O título (e a proposta como um todo) poderia ser um pouco mais geral focando na fronteira irregular e no impacto em agentes LLM, deixando a tarefa de engenharia de poços como estudo de caso. Não é claro para mim o quanto essa tarefa difere de outras, justificando que a tese seja tão específica assim. O contexto de agentes baseados em LLM é bem amplo, e entender essa nova classe de problemas sob a perspectiva da fronteira irregular me parece mais interessante. O segundo ponto que me chama a atenção no título é ``análise experimental" que novamente é restritivo. Acho que tem que ter alguma formulação teórica, um arcabouço, uma lógica mais }

\section{Introdução e Contextualização}
\label{introduuxe7uxe3o-e-contextualizauxe7uxe3o}



A evolução recente dos modelos de linguagem marca uma transição da IA concebida predominantemente como ferramenta de consulta pontual para sua configuração como agente autônomo \cite{ferrag2025review} \cite{ibm2024agents_} \wagner{ref?} \vitor{feito}. Nessa nova configuração, sistemas de agentes são capazes de decompor problemas, planejar e executar tarefas de múltiplos passos, articulando raciocínio, memória e uso de ferramentas externas \cite{yao2023react} \cite{lewis2020retrieval}  \wagner{ref?} \vitor{feito}. Na prática, essa transição se materializa em soluções comerciais de agentes autônomos, como Manus AI \footnote{https://www.manus.ai/}, OpenAI Operator \footnote{https://openai.com/pt-BR/index/introducing-operator/}, OpenAI Deep Research \footnote{https://openai.com/pt-BR/index/introducing-deep-research/} e Genspark AI \footnote{https://www.genspark.ai/}.

\wagner{Aqui podem ser referências ou footnotes, o que for mais prático. Por exemplo, se forem só as URLs, melhor que seja footnote.} \vitor{feito}, entre outras.

\wagner{As referências \textbf{tem} que estar no texto usando cite ou citep. Mais ainda, use o bibtex para evitar ter referências desformatadas ou incompletas.}
\vitor{feito}

Contudo, a rápida adoção dessa tecnologia na indústria, especialmente em setores de alto risco como Óleo e Gás (O\&G), supera nossa compreensão de seus reais limites \wagner{acho que esse tom de O\&G como cenário típico fica melhor, mas precisamos de referências que falem desse assunto}. O entusiasmo com as capacidades em tarefas ``dentro da fronteira'' (\emph{inside the frontier}) muitas vezes ofusca a existência de tarefas ``fora da fronteira'' (\emph{outside the frontier}), onde a IA falha \cite{dell2023navigating}. 
\wagner{Algum artigo nisso? Senão podemos converter em sub-produto do trabalho a caracterização de picos e vales}
\vitor{Nenhum artigo com essa caracterização. Acho melhor usarmos a mesma nomeclatura do artigo da HBS, "inside the frontier" e "outside the frontier". Fiz o ajuste no projeto. }

\cite{dell2023navigating} introduziu o conceito de ``Fronteira Tecnológica Irregular'' (Jagged Frontier) para descrever como o desempenho da IA é irregular, alternando entre tarefas ``dentro da fronteira'' (\emph{inside the frontier}) e tarefas ``fora da fronteira'' (\emph{outside the frontier}). No entanto, este estudo focou na produtividade de \emph{humanos usando IA} em tarefas de consultoria.

A lacuna que este projeto busca abordar é a falta de conhecimento a respeito da dita fronteira irregular para agentes autônomos em domínios de engenharia de alta complexidade. 
\wagner{tem mais algum além de engenharia de poços? quais as refs?} 
\vitor{inserido abaixo}
Embora estudos recentes tenham documentado limitações similares em outras áreas críticas como medicina \cite{huang2025medical, hatem2023hallucinations}, direito \cite{li2024lawyer, gogani2025tax}, engenharia civil \cite{frontiers2025civil, zheng2025aec}, finanças \cite{chen2024financial} e sistemas \emph{safety-critical} em aeroespacial \cite{nasa2025llm, liu2024safetycritical}, não se sabe que tipo de tarefas compreendidas na construção de poços de petróleo (um domínio em que falhas podem frequentemente custar vidas e/ou prejuízos materiais relevantes) estariam ``dentro da fronteira'' (\emph{inside the frontier}) e quais estariam ``fora da fronteira'' (\emph{outside the frontier}) de performance e assertividade das ferramentas.
\wagner{aqui acho que vale explicar um pouco mais o contexto de uso e como tarefas dentro e fora da fronteira se manifestariam no problema de engenharia de poços.}

\wagner{um exemplo de como agentes LLM são usados em engenharia de poços seria bem ilustrativo}

\section{Problema de Pesquisa e
Objetivos}\label{problema-de-pesquisa-e-objetivos}


\subsection{Problema Central}\label{problema-central}

A implantação de agentes de LLM em atividades diversas de O\&G é
dificultada pela falta de um mapa de risco-capacidade. As métricas de
\emph{benchmarks} genéricos (ex: MMLU, AgentBench) não capturam as
nuances de tarefas de engenharia do mundo real, que envolvem dados
ruidosos, raciocínio físico e adesão estrita a normas de segurança.
\wagner{Acho que está muito específico, vou tentar reescrever, mas apenas para ilustrar o que seria mais abstrato. Entendo que conceitos como mapa risco capacidade são genéricos e aplicados a vários problemas.}

\wagner{O uso de agentes de LLM em atividades diversas, incluindo missões críticas em engenharia, é dificultada pela falta de um mapa risco-capacidade. \emph{Benchmarks} genéricos não capturam nuances do mundo real, que envolvem dados ruidosos, raciocínio físico e adesão estrita a normas de segurança.}


\subsection{Pergunta Principal de Pesquisa
(P1)}\label{pergunta-principal-de-pesquisa-p1}

Onde se localiza, e qual é a topografia, da ``fronteira irregular'' de
capacidade para agentes de LLM no domínio de planejamento e execução de
tarefas da construção de poços offshore?

\wagner{Achei a pergunta central de pesquisa muito específica. Vou tentar ampliar novamente}

\wagner{A hipótese principal deste trabalho é que é possível identificar a fronteira irregular de capacidade para agentes LLM e caracterizar a sua  localização e topografia em domínios relevantes e significativos, em particular planejamento e execução de tarefas de engenharia.}

\subsection{Perguntas Secundárias (P2-P4)}\label{perguntas-secunduxe1rias-p2-p4}

\wagner{Achei as perguntas secundárias bem niveladas}
\begin{itemize}
\item
  \textbf{(P2)} Quais características de uma tarefa (ex: necessidade de
  raciocínio causal, dependência de dados físicos, conformidade
  regulatória, planejamento temporal) definem se ela está ``dentro da fronteira'' (\emph{inside the frontier} -- sucesso do
  agente) ou ``fora da fronteira'' (\emph{outside the frontier} -- falha do agente)?
\item
  \textbf{(P3)} Como diferentes arquiteturas de agentes (ex: LLM
  ``puro'' vs.~RAG vs.~Agentes de Planejamento) navegam por essa
  fronteira?
\item
  \textbf{(P4)} É possível desenvolver um \emph{framework} para
  identificar tarefas ``fora da fronteira'' \emph{a priori}, permitindo a implantação segura
  de agentes em tarefas ``dentro da fronteira''?
\end{itemize}

\wagner{Acrescentaria uma pergunta da aplicação desse framework em cenários relevantes e significativos}

\subsection{Objetivo Geral}\label{objetivo-geral}

Mapear e caracterizar a fronteira irregular de capacidade de agentes de
LLM no domínio de engenharia de poços, identificando os fatores que
determinam o sucesso e a falha em tarefas complexas.

\wagner{Ficou meio parecido com a minha proposta de problema de pesquisa (me desculpe). Eu iria além de engenharia de poços.}

\subsection{Objetivos Específicos}\label{objetivos-especuxedficos}

\begin{enumerate}
\def\labelenumi{\arabic{enumi}.}
\item
  \textbf{OE1:} Desenvolver uma taxonomia de tarefas representativas da
  construção de poços offshore, classificadas por tipo de cognição e
  complexidade.
\item
  \textbf{OE2:} Projetar e implementar um \emph{benchmark} experimental
  baseado nesta taxonomia, com métricas de avaliação e \emph{ground
  truth} definidos por especialistas.
\item
  \textbf{OE3:} Avaliar sistematicamente diferentes arquiteturas de
  agentes de LLM neste \emph{benchmark}.
\item
  \textbf{OE4:} Analisar os resultados para construir o ``mapa'' da
  fronteira irregular, correlacionando tipos de tarefa com o desempenho
  dos agentes.
\item
  \textbf{OE5:} Propor um \emph{framework} de decisão para a implantação
  segura de agentes na indústria de O\&G, baseado nas descobertas.
\end{enumerate}


\section{Justificativa e Relevância}\label{justificativa-e-relevuxe2ncia}

Este projeto possui relevância em três eixos:

\begin{enumerate}
\def\labelenumi{\arabic{enumi}.}

\item
  \textbf{Contribuição para a Ciência da Computação (Teórica):} Estende
  a teoria da ``Fronteira Irregular'' do campo de Interação
  Humano-Computador (HCI) para o campo de Agentes Autônomos. Além disso,
  critica e avança o estado da arte em \emph{benchmarking} de agentes,
  saindo de tarefas genéricas para domínios industriais complexos.
\item
  \textbf{Contribuição para a Indústria de O\&G (Prática):} Fornece o
  primeiro estudo rigoroso sobre o que agentes de IA podem (e,
  crucialmente, \emph{não podem}) fazer com segurança na engenharia de
  poços. Isso desbloqueia ganhos de eficiência (em tarefas ``dentro da fronteira'') e previne
  falhas catastróficas (em tarefas ``fora da fronteira'').
\item
  \textbf{Originalidade:} A intersecção de Agentes LLM, a teoria da
  ``Jagged Frontier'' e o domínio de O\&G \emph{onshore/offshore} é
  inteiramente nova na literatura.
\end{enumerate}

\section{Fundamentação Teórica}\label{fundamentauxe7uxe3o-teuxf3rica}

A tese será fundamentada em quatro pilares:

\begin{enumerate}
\def\labelenumi{\arabic{enumi}.}
\item
  \textbf{Agentes Baseados em LLM:} Arquiteturas e paradigmas (RAG,
  ReAct, CoT, Multi-Agentes). Como eles funcionam, planejam e usam
  ferramentas.
\item
  \textbf{Avaliação de Agentes (Benchmarking):} Estado da arte (ex:
  AgentBench, GAIA, MT-Bench). Análise de suas limitações para tarefas
  industriais/engenharia.
\item
  \textbf{Produtividade e Limites da IA:} O \emph{paper} seminal de
  Dell'Acqua et al.~(2023) sobre a ``Fronteira Irregular''.
\item
  \textbf{Engenharia de Poços e IA:} Aplicações atuais de machine
  learning em O\&G e a lacuna existente na aplicação de \emph{agentes
  generativos} para atividades diversas do setor.
\end{enumerate}


\section{Metodologia Proposta}\label{metodologia-proposta}

Este projeto empregará uma \textbf{metodologia de pesquisa experimental
quantitativa e qualitativa}, dividida em quatro fases:

\textbf{Fase 1: Definição do Domínio e Taxonomia de Tarefas (OE1)}

\begin{itemize}
\item
  \textbf{Fonte de Dados:} Análise documental de Normas Técnicas,
  Padrões Operacionais, Relatórios de Situação Operacional, Lições
  Aprendidas, Alertas Técnicos e Relatórios Diários de Perfuração
  (DDRs/Boletins Diários de Operação - BDOs).
\item
  \textbf{Amostragem:} Criação de um \emph{dataset} de 20-40 tarefas
  representativas.
\item
  \textbf{Classificação (Taxonomia):} As tarefas serão classificadas por
  eixos:

  \begin{itemize}
  \item
    \emph{Tipo de Ação:} Extração de Informação, Síntese, Diagnóstico,
    Planejamento, Verificação de Conformidade.
  \item
    \emph{Domínio de Conhecimento:} Geologia, Fluidos, Mecânica,
    Regulação.
  \item
    \emph{Complexidade:} Nível de raciocínio causal, temporal e espacial
    exigido.
  \end{itemize}
\end{itemize}

\textbf{Fase 2: Design do Benchmark Experimental (OE2)}

\begin{itemize}
\item
  \textbf{Plataforma:} Desenvolvimento de um ambiente de teste (sandbox)
  onde os agentes podem atuar.
\item
  \textbf{Ferramentas (Tools):} Disponibilização de ``ferramentas''
  simuladas para os agentes (ex: \texttt{buscar\_norma\_api(id)},
  \texttt{calcular\_volume\_anular(diametros)},
  \texttt{ler\_ultimo\_ddr()}).
\item
  \textbf{Ground Truth:} Definição de critérios de sucesso (o
  ``gabarito'') para cada tarefa, validado por Especialistas no Domínio
  (SMEs - \emph{Subject Matter Experts}).
\end{itemize}

\textbf{Fase 3: Execução Experimental (OE3)}

\begin{itemize}
\item
  \textbf{Variáveis Independentes:} Arquitetura do Agente.
\item
  \textbf{Variáveis Dependentes (Métricas):}

  \begin{enumerate}
  \def\labelenumi{\arabic{enumi}.}
  \item
    \emph{Taxa de Sucesso Binário:} Completou a tarefa com sucesso?
  \item
    \emph{Qualidade da Resposta:} Avaliação cega (1-5) por SMEs.
  \item
    \emph{Eficiência:} Custo (tokens), passos de raciocínio.
  \item
    \emph{Robustez:} O agente ``alucina'' ou falha?
  \end{enumerate}
\end{itemize}

\textbf{Fase 4: Análise e Mapeamento da Fronteira (OE4, OE5)}

\begin{itemize}
\item
  \textbf{Análise Quantitativa:} Correlação estatística entre as
  \emph{características da tarefa} (da Fase 1) e as \emph{métricas de
  desempenho} (da Fase 3).
\item
  \textbf{Análise Qualitativa:} Análise de causa-raiz das falhas
  (tarefas ``fora da fronteira''). O agente falhou por falta de conhecimento (RAG falho),
  raciocínio (LLM falho) ou planejamento (arquitetura falha)?
\item
  \textbf{Resultado:} O ``mapa'' da fronteira e o \emph{framework} de
  decisão.
\end{itemize}

\section{Cronograma Preliminar (48 meses)}\label{cronograma-preliminar-48-meses}

\begin{itemize}
\item
  \textbf{Ano 1 (Meses 1-12):}

  \begin{itemize}
  \item
    Revisão de Literatura aprofundada.
  \item
    Disciplinas obrigatórias.
  \item
    Execução da Fase 1 (Taxonomia de Tarefas).
  \item
    Definição do projeto de tese (Exame de Qualificação).
  \end{itemize}
\item
  \textbf{Ano 2 (Meses 13-24):}

  \begin{itemize}
  \item
    Execução da Fase 2 (Desenvolvimento do Benchmark).
  \item
    Testes-piloto.
  \item
    Artigo de revisão ou \emph{position paper} sobre o \emph{benchmark}.
  \end{itemize}
\item
  \textbf{Ano 3 (Meses 25-36):}

  \begin{itemize}
  \item
    Execução da Fase 3 (Bateria principal de testes e avaliação).
  \item
    Coleta de dados (avaliação pelos SMEs).
  \item
    Início da Fase 4 (Análise).
  \item
    Submissão de artigo para conferência principal (ex: NeurIPS, ICML,
    ou conferência de O\&G como a OTC).
  \end{itemize}
\item
  \textbf{Ano 4 (Meses 37-48):}

  \begin{itemize}
  \item
    Conclusão da Fase 4 (Desenvolvimento do Framework).
  \item
    Redação da Tese.
  \item
    Submissão de artigo em periódico.
  \item
    Defesa.
  \end{itemize}
\end{itemize}

\section{Referências Bibliográficas Preliminares}\label{referuxeancias-bibliogruxe1ficas-preliminares}

\wagner{Coloquei a primeira ref em  vitor.bib e citei no texto. Acho que deve ser feito com todas e esse itemize e mesmo essa seção perdem sentido}


\begin{itemize}
\item
  Dell'Acqua, Fabrizio, et al.~(2023). \emph{Navigating the Jagged
  Technological Frontier: Field Experimental Evidence of the Effects of
  AI on Knowledge Worker Productivity and Quality}. Harvard Business
  School.
\item
  Yao, S., et al.~(2023). \emph{ReAct: Synergizing Reasoning and Acting
  in Language Models}.
\item
  Lewis, P., et al.~(2020). \emph{Retrieval-Augmented Generation for
  Knowledge-Intensive NLP Tasks}.
\item
  Zeng, Y., et al.~(2024). \emph{AgentBench: Evaluating LLMs as Agents}.
\item
  Manus AI. (2024). \emph{Manus AI: Autonomous AI agents for complex
  workflows}. Recuperado de https://www.manus.ai.
\item
  OpenAI. (2024). \emph{OpenAI Operator: Building and orchestrating
  AI-native applications}. Recuperado de https://platform.openai.com.
\item
  OpenAI. (2024). \emph{Deep Research: Autonomous research agent by
  OpenAI}. Recuperado de https://platform.openai.com.
\item
  Genspark AI. (2024). \emph{Genspark AI: Autonomous AI agents for
  research and knowledge work}. Recuperado de https://www.genspark.ai.
\end{itemize}
